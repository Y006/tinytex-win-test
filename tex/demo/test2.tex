% ===================================================================
% 文件名: tex/demo/main.tex
% 用途: TinyTeX 全栈环境自检报告 (不依赖 elegantpaper)
% ===================================================================

% [Group 2] 关键:强制指定 Fandol 字体,兼容 Windows Server
\PassOptionsToPackage{fontset=fandol}{ctex}

% [Group 1] 使用标准 article 类,测试基础 tools
\documentclass[a4paper, 12pt]{article}

% ===================================================================
% 0. 宏包加载区 (对应 install-tinytex.ps1 的 5 个分组)
% ===================================================================

% --- [Group 2] 中文支持 ---
\usepackage{ctex}      % 中文核心
\usepackage{zhnumber}  % 中文数字 (一、二、三)

% --- [Group 3] 字体与数学 ---
\usepackage{iftex}     % 引擎检测
\usepackage{amsmath, amsfonts, amssymb} % 数学三剑客
\usepackage{esint}     % 扩展积分符号
% newtx 会改变默认字体 (Times 风格),验证字体包是否安装
\usepackage{newtxtext, newtxmath} 

% --- [Group 4] 模板逻辑与底层工具 ---
\usepackage{appendix}  % 附录支持
\usepackage{abstract}  % 摘要定制
\usepackage{hologo}    % TeX Logo 显示 (如 \XeTeX)
\usepackage[stable]{footmisc} % 脚注增强
\usepackage{etoolbox}  % 编程逻辑工具

% --- [Group 5] 格式与功能 (重头戏) ---
\usepackage[margin=2.5cm]{geometry} % 页面尺寸
\usepackage{xcolor}    % 颜色支持
\usepackage{graphicx}  % 图片支持
\usepackage{mwe}       % 测试用图片资源 (example-image)
\usepackage{booktabs}  % 专业三线表
\usepackage{multirow}  % 表格合并行
\usepackage{makecell}  % 表格单元格换行
\usepackage{caption}   % 标题定制
\usepackage{subcaption}% 子图支持
\usepackage{enumitem}  % 列表定制
\usepackage{listings}  % 代码高亮
\usepackage{fancyvrb}  % 增强 verbatim
\usepackage{siunitx}   % 物理单位
\usepackage{microtype} % 排版微调
\usepackage{tikz}      % 绘图底层 (pgf)
\usepackage[colorlinks=true, linkcolor=blue]{hyperref} % 超链接
\usepackage{cleveref}  % 智能引用 (必须放在 hyperref 之后)

% --- [Group 5] 参考文献 (使用 biblatex) ---
\usepackage[style=gb7714-2015, backend=biber]{biblatex} % 调用 gbt7714 样式

% ===================================================================
% 1. 准备测试数据
% ===================================================================
% 自动生成参考文献数据
\begin{filecontents}[overwrite]{ref.bib}
@manual{tinytex,
  title  = {TinyTeX 文档},
  author = {Yihui Xie},
  year   = {2024},
  url    = {https://yihui.org/tinytex/}
}
@article{latex3,
  title   = {The LaTeX3 Interfaces},
  author  = {The LaTeX3 Project},
  journal = {TUGboat},
  year    = {2023}
}
\end{filecontents}
\addbibresource{ref.bib}

% TikZ 设置
\usetikzlibrary{shapes.geometric, arrows}

% 代码块设置
\lstset{
    basicstyle=\ttfamily\small,
    backgroundcolor=\color{gray!10},
    frame=single,
    keywordstyle=\color{blue}
}

% ===================================================================
% 正文开始
% ===================================================================
\title{\textbf{TinyTeX 环境全栈自检报告}}
\author{CI/CD Pipeline}
\date{\zhtoday} % 测试 ctex 的日期功能

\begin{document}

\maketitle

\begin{abstract}
    \noindent 
    本报告完全脱离 \texttt{elegantpaper} 等外部模板,仅使用标准 \texttt{article} 类,
    旨在验证 \texttt{install-tinytex.ps1} 脚本中定义的 5 个分组宏包是否全部安装成功并能正常协同工作。
    如果本文档编译无误,说明环境已具备处理中文、数学、绘图、代码及参考文献的全部能力。
\end{abstract}

\tableofcontents
\vspace{1cm}

% -------------------------------------------------------------------
\section{核心环境与中文支持 (Group 1 \& 2)}
% -------------------------------------------------------------------
\subsection{编译引擎检测 (iftex)}
当前编译引擎为:
\ifxetex
    \textcolor{green!60!black}{\textbf{\hologo{XeTeX}}} (检测通过,符合预期)
\else
    \textcolor{red}{\textbf{未知引擎}} (检测失败)
\fi

\subsection{中文与字体 (ctex + fandol + zhnumber)}
\begin{itemize}
    \item \textbf{基础中文}:汉字显示正常,说明 \texttt{ctex} 宏包及 Fandol 字体集已正确加载。
    \item \textbf{中文数字}:调用 \texttt{zhnumber} 包,数字 2025 转换为:\textbf{\zhnumber{2025}}。
    \item \textbf{字体样式}:\textit{Italic 样式 (Fandol楷体)},\textbf{Bold 样式 (Fandol黑体)}。
\end{itemize}

% -------------------------------------------------------------------
\section{数学与字体 (Group 3)}
% -------------------------------------------------------------------
本节测试 \texttt{newtx} 字体包与 \texttt{ams*} 系列宏包。
\begin{equation}
    \label{eq:maxwell}
    \oint_{\partial \Omega} \mathbf{E} \cdot d\mathbf{l} = -\frac{d}{dt} \iint_S \mathbf{B} \cdot d\mathbf{S}
\end{equation}
\textbf{测试点}:
\begin{enumerate}[label=\alph*)]
    \item 公式字体应为 Times 风格 (由 \texttt{newtxmath} 提供)。
    \item 闭合积分号 $\oint$ 显示正常 (由 \texttt{esint} 或基础包提供)。
    \item 智能引用测试:参见 \Cref{eq:maxwell} (由 \texttt{cleveref} 提供)。
\end{enumerate}

% -------------------------------------------------------------------
\section{多媒体与绘图 (Group 5)}
% -------------------------------------------------------------------
\subsection{智能子图 (caption + subcaption + mwe)}
测试 \texttt{mwe} 包提供的占位图以及子图排版功能。

\begin{figure}[htbp]
    \centering
    \begin{subfigure}{0.45\textwidth}
        \includegraphics[width=\linewidth]{example-image-a}
        \caption{测试图 A}
    \end{subfigure}
    \hfill
    \begin{subfigure}{0.45\textwidth}
        \includegraphics[width=\linewidth]{example-image-b}
        \caption{测试图 B}
    \end{subfigure}
    \caption{子图功能测试}
\end{figure}

\subsection{矢量绘图 (pgf/tikz)}
测试 \texttt{pgf} 底层库是否能绘制图形。

\begin{center}
\begin{tikzpicture}
    % 画一个流程图节点
    \node (start) [rectangle, rounded corners, minimum width=3cm, minimum height=1cm, text centered, draw=black, fill=red!30] {开始测试};
    \node (process) [rectangle, minimum width=3cm, minimum height=1cm, text centered, draw=black, fill=orange!30, below of=start, node distance=2cm] {加载宏包};
    \node (end) [rectangle, rounded corners, minimum width=3cm, minimum height=1cm, text centered, draw=black, fill=green!30, below of=process, node distance=2cm] {编译成功};
    
    \draw [->, thick] (start) -- (process);
    \draw [->, thick] (process) -- (end);
\end{tikzpicture}
\end{center}

% -------------------------------------------------------------------
\section{复杂表格与代码 (Group 5)}
% -------------------------------------------------------------------
\subsection{专业表格 (booktabs + multirow + makecell)}
\begin{table}[htbp]
    \centering
    \caption{宏包安装情况概览}
    \begin{tabular}{llcc}
        \toprule
        \textbf{分组} & \textbf{核心包} & \textbf{状态} & \textbf{备注} \\
        \midrule
        Group 1 & latexmk & \textcolor{blue}{OK} & 自动化 \\
        Group 2 & \makecell[l]{ctex\\xecjk} & \textcolor{blue}{OK} & 中文 \\
        \multirow{2}{*}{Group 5} & tikz & \textcolor{blue}{OK} & 绘图 \\
                                 & biblatex & \textcolor{blue}{OK} & 引用 \\
        \bottomrule
    \end{tabular}
\end{table}

\subsection{代码高亮 (listings)}
测试代码环境及 \texttt{siunitx} 单位显示:
\begin{lstlisting}[language=TeX, caption=LaTeX 代码示例]
% 物理单位测试
重力加速度: \qty{9.8}{\meter\per\second\squared}
光速: \qty{2.99e8}{\meter\per\second}
\end{lstlisting}
渲染结果:重力加速度: \qty{9.8}{\meter\per\second\squared}。

% -------------------------------------------------------------------
\section{底层逻辑与引用 (Group 4 \& 5)}
% -------------------------------------------------------------------
\subsection{编程逻辑 (etoolbox)}
\textbf{逻辑测试}:
\ifdef{\maketitle}
    {命令 \texttt{\textbackslash maketitle} 已定义 (etoolbox 工作正常)。}
    {错误:命令未定义。}

\subsection{参考文献 (biblatex + gbt7714)}
本文引用了 TinyTeX 手册\cite{tinytex}以及 LaTeX3 项目\cite{latex3}。
请检查文末是否生成了符合 GB/T 7714 标准的参考文献列表。

% 打印参考文献
\printbibliography[heading=bibintoc, title=参考文献]

% -------------------------------------------------------------------
\appendix
\section{附录 (appendix)}
% -------------------------------------------------------------------
本附录用于测试 \texttt{appendix} 宏包。
如果能看到这里的 "A 附录",说明文档结构完整。

\end{document}